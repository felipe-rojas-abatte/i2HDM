\message{ !name(extending_limits.tex)}\documentclass[12pt,a4paper]{article}
\hoffset=-2cm 
\textwidth=17.5cm
\usepackage{amsmath}
\usepackage{amssymb}

\usepackage[affil-it]{authblk}

\usepackage[utf8]{inputenc}
\usepackage{url}
\usepackage{hyperref}
\usepackage{graphicx}
\usepackage{xcolor}


%%%%%%%%%%%%%%%%%%%%%%%
%%%%%%%%%%%%%%%%%%%%%%%


\begin{document}

\message{ !name(extending_limits.tex) !offset(-3) }


\title{Projecting 8 TeV checkmate limits to 13 TeV.}
\maketitle
Our 95\% CL limits from checkmate with 10 $fb^{-1}$ at 8 TeV is $\sim$ 2 $pb$.

To project to 13 TeV, and for different luminosities, we need to find how the background changes, $B \to B^{\prime}$, for new energies/luminosities, and then we can find how the signal must change, $S \to S^{\prime}$, such that the significance, $\left(\tfrac{S}{\sqrt{B}} \right)$ is unchanged (at the level giving a 95\% CL).

i.e. 

\begin{equation}
  \tfrac{S}{\sqrt{B}} = \tfrac{S^{\prime}}{\sqrt{B^{\prime}}}
\end{equation}
where $B$ is the original background, $B^{\prime}$ is the background at the new enrgy/luminosity, $S$ is the original signal ruled out at eh 95\% CL and $S^{\prime}$ is the new signal ruled out at a 95\% CL.

In our case, we have that with 10 $fb^{-1}$ of integrated luminosity,\\
$\sigma_{8 TeV} = 8 $ fb\\
$\sigma_{13 TeV} = 50$ fb\\

Assuming the background scales in a similar manner

\message{ !name(extending_limits.tex) !offset(-5) }

\end{document}


