\documentclass[11pt]{article}

\usepackage{xcolor} 
\usepackage{ulem}
\usepackage{soul}
\usepackage{amsmath}
\usepackage{graphicx}
\usepackage{dcolumn}
\usepackage{hyperref}
\usepackage{bm}
\usepackage{bbm}
\usepackage{epsfig}
\usepackage{slashed}
\usepackage{amssymb}
\usepackage{color}
\usepackage[font=small]{caption}
\usepackage[font=small]{subcaption}
%\usepackage{tabulary}
\voffset=-2.5cm
\hoffset=-2.cm
\textwidth=16.5cm
\textheight=23.5cm

\newcommand{\tcb}[1]{\textcolor{blue}{#1}}
\newcommand{\tco}[1]{ {\color{orange}{#1}} }
\newcommand{\tcm}[1]{\textcolor{magenta}{#1}}
\newcommand{\tcr}[1]{\textcolor{red}{#1}}

 


%%%%%%%%%%%%%%%%%%%%%%%%%%%%%%%%%%%%%%%%%%%%%%%%%%%%%%%
\begin{document}

\noindent 
Dear editor, 
\\
\\
We would like to thank the referee for the detailed comments and interest to our paper,
which helped us to improve our paper.
We deeply apologise for the delay with our reply,
and hope that the new version of the paper is suitable for the publication in PRD.
Our reply and summary of modifications is below.
\\
\\
\noindent Sincerely, 
\\
\\
\noindent 
Alexander Belyaev on behalf of authors
\\
\\
\\
{\bf (1) Referee comment:} { \it \\
In Ref. [29], they claim that the perturbativity and unitarity is
more reliable by using the code 2HDMC than Ref. [51] and [60],
especially for HHHH vertex. However, since the author does not use
2HDMC instead of the conditions presented in Ref. [51] and [60], I
suggest the authors shall also check whether is there some parameter
space not properly tested.
}\\

{\bf Response:} 
We have explicitly checked that the limit
from perturbative unitarity and perturbativity given by Eq.(13)-(16) we are using in our study is
consistent with that implemented in the 2HDMC code[68].
We have added the respective reference and sentence to an updated version of the paper.
Moreover, we would like to note that in our paper we present the limit on $\lambda_2$ (or $\lambda_{HHHH}$  -- notation from [29])
as a function of $\lambda_{345}$ pictorially presented in Figure 2 which goes beyond findings in [29].

\bigskip

\noindent
{\bf (2) Referee comment:} {\it  \\
"A related question is whether the model can be better probed by
indirect detection (ID) experiments, i.e. the detection of energetic
cosmic rays like positron, gamma ray, and antiproton, which may be
created by the annihilation of h1 pairs. ..., but the bounds are not
competitive with those coming from DD."

Recently, the new AMS02 antiproton data seems to have some interesting
behavior, see Ref. [1610.03071] and [1610.03840]. This could put some
significant impact on i2HDM parameter space. I suggest the authors can
add some discussion and estimation about this.
}

\bigskip

{\bf Response:}

We have checked that the strongest bounds on the i2HDM
parameter space coming from such experiments are set by gamma ray telescopes: both the Fermi-
LAT gamma-ray space telescope [84] as well as ground based telescopes. Fermi-LAT is sensitive
to gamma rays particularly in the low mass range up to O(100 GeV), but the bounds are not
competitive with those coming from DD. This conclusion is also confirmed by studies in Ref. [60].

In what concerns the AMS02 results on antiprotons, given the existing uncertainties
in the secondary anti-proton production, we do not take it as a robust indication of the a DM signature. 
Even if assumed to be a real signal, incorporating it into our picture would bring novel uncertainties 
of purely astrophysical origin such as the DM distribution profiles.  
Thus, it cannot be easily translated into a new constraint on the i2HDM parameter space.
We prefer to stay conservative and do not include these data in our analysis.

The paper was updated respectively with this sentence.

\bigskip
%%%%%%%%%%%%%%%%%%%%%%%%%%%%%%%%%%%%%%%%%%%%%%%%%%%%%%%%%%%%%%%%%%%%%%%%%
\noindent
{\bf (3) Referee comment:} {\it
The new allowed region (almost mh1~mh2 and la345~0) discovered by
the authors shall be identified and more emphasized.

I was interesting in this region and trying to see how this region can
be survived but in the end I only found the BM1 at the Table 1. Since
this is the new region claimed by the authors, some more information
shall be given.

I suspect the relic density of most this region are always too low
from PLANCK data. For example, the BM1 has the relic density 0.092 and
$\chi^2$ based on Eq. (29) are 484 which is not good for thermal relic
scenario! If my guess is true, the authors shall mention whether is
this new region owing to the non-thermal relic assumption?

}
\bigskip
{\bf Response:}
In the new version of the paper we have 
provided the requested details on this point.

First of all, we made clear in the end of page 14 that 
in our analysis we have assumed 10\% theoretical uncertainty  on the DM relic density prediction since its  based on the tree-level calculation. This uncertainty is the  dominant one in comparison to about 1\% uncertainty on DM fit from the latest PLANCK results given above and relax the DM relic density limit 
to the following one at 95\% CL:
\begin{equation}
\Omega_{\rm DM}^{\rm limit} h^2=0.1184\pm 2\times 0.1184/10
\simeq 0.118\pm 2\times 0.012
\label{eq:planck-limit-relaxed}
\end{equation}

Then in section 3.2.1 we discuss all details of this region,
pointing to:
a)that the width of this strip is defined by 
the maximum allowed value of $\Delta M = M_{h_2}-M_{h_1}=8$~GeV, above which
the parameter space is excluded by LEP di-lepton searches
b) we state that 
in this allowed region DM relic density is never above PLANCK limit given by Eq.~\ref{eq:planck-limit-relaxed};
c) that
the maximum value of $\Omega h^2_{DM}$ reaches
the value of about 0.11 for  $M_{h_2}-M_{h_1} \simeq 8$~GeV
and $\lambda_{345}\simeq 0$, when the only 
${h}_1-{h}_2$ co-annihilation takes place;
d)
for  $\Delta M<8$~GeV
and $M_{h_1}<54$~GeV, the $\Omega h^2$ drops below the $0.118-2\times0.012$ limit which use in our study
because ${h}_1-{h}_2$ co-annihilation via $Z$-boson increases 
with  he decrease of ${h}_1,{h}_2$ masses;
e)
the upper edge
at $73$ GeV is defined by the rapid increase of the  $h_1 h_1 \to W W^*$ contribution,
which does not require co-annihilation above this mass. The typical $M_{h_2}-M_{h_1}$
mass split in the co-annihilation region is 7-8 GeV, is required to make the relic density 
consistent with the PLANCK limit.

To conclude on this point,  taking into account theoretical uncertainty  on the DM relic density prediction, the relic density in this region is consistent with PLANCK limit and reaches 0.11 value at maximum.
This region  therefore does not necessarily require non-thermal scenario,
which in case of weak interactions would be a bit contrived.

\bigskip

%%%%%%%%%%%%%%%%%%%%%%%%%%%%%%%%%%%%%%%%%%%%%%%%%%%%%%%%%%%%%%%%%%%%%%%%%
\noindent
{\bf (4) Referee comment:} {\it
Again, I was so looking forward to seeing the result about the life
time of h+ and the limit of delta M in the last paragraph of Section
3. However, the authors disappoint me by only using one paragraph with
a lot of hand-waving. I think the authors shall present this part with
useful figure(s). Otherwise, it is hard to drive me trust the sentence
written in the conclusion:

"At the same time, the potential of the LHC using a search for
disappearing charged tracks is quite impressive in probing Mh1 masses
up to about 500 GeV already at 8 TeV with 19.5 fb-1 luminosity as we
have found in our study."
}

\bigskip

{\bf Response:}
In the updated  version of the paper have provided all requested details in 
section 3.2.2:
\\
a) details of the evolution of the $h^+$ decay width in case of $h^+-h_1$ degeneracy,
which include effective $W-pion$ mixing;
\\
b) we provide the formula for the decay width of $h^+$;
\\
c) we provide plots for decay width and the charged track length for $h^+$;
(Fig.10)
\\
d) we give the cross section for  $h^+$ production and the relic density
for the relevant 400-600 GeV mass range





%%%%%%%%%%%%%%%%%%%%%%%%%%%%%%%%%%%%%%%%%%%%%%%%%%%%%%%%%%%%%%%%%%%%%%%%%

\bigskip
\noindent
{\bf (5) Referee comment:} {\it  \\
Other suggestions:
\\
(i) The text is well organized and very enjoyably readable only before
section 3

but it becomes messy and hard to read after section 4. Especially, in
the result section, many information/concepts/reasons are given
scatteringly and all the figures are listed in the back. I believe
that the authors should be able to make it much clear in the
presentation. For example, they can use bullets, items, and tables for
classification and comparison.

In addition, I strongly suggest the authors to locate their figures
before the text. The figure location can also help reader. Some less
important figures can be integrated into an appendix.
\\
\\
(ii) Missing the space (Fig.20, Fig.21)
}
\bigskip

{\bf Response:} 
We appreciate very much the Referee remark about the structure of the paper 
and completely agree with this point.
In the new version of the paper we have done the the following changes:
\\
a)  we have moved many figures on the scan of the parameter space into appendix, this making paper easier to read;
\\
b) we have produced simplified version of figures -- Fig.6-8
for the section 3,  aiming the same purpose -- to make it easier readable.

\bigskip

To conclude -- we agree with all remarks of the Referee
and implemented the respective changes which include   more details on the long-lived 
$h^+$ case as well as restructuring the paper. We believe these changes improve the quality of the paper and thank referee for this.
We hope that the new version of the paper is now suitable for the publication in PRD.

\end{document}
