\subsection{Constraints from LHC Higgs data}

The LHC Higgs data further restricts the i2HDM parameters space in the form of constraints on the couplings of the SM-like Higgs boson. A collection of combined fits from the Run I data, for both ATLAS and CMS, can be found in~\cite{Khachatryan:2016vau}.
In the i2HDM, the leading effect is encoded in two observables: the decays of the Higgs into two Dark Matter scalars, $H \to h_1 h_1$, which is kinematically open when $m_{h_1} < M_H/2$; and the contribution of the charged Higgs loops to the $H \to \gamma \gamma$ decay. 
In principle, we would need to do a two-parameter fit of the available Higgs data.  None of the fits presented in~\cite{Khachatryan:2016vau} can therefore be directly applied in our case.

A simpler possibility is, instead, to consider the best possible bound from the available fits on the two parameters.
We follow this simpler procedure, confident that it will lead to a somewhat more conservative estimation of the bounds.
For the invisible Higgs branching ratio, we consider the bound coming from the dedicated ATLAS search~\cite{Aad:2015txa}
\begin{equation}
Br(H\to invisible)  < 28\% 
\label{eq:lhc-higgs-invis}
\end{equation}
at the 95$\%$ CL, which is comparable with a 36\% limit
from the combined CMS analysis~\cite{CMS:2015naa}.\footnote{One could also limit $Br(H\to invisible)$ 
using $Br(H\to BSM)<34\% $ at 95$\%$CL exclusion from Run1  ATLAS-CMS Higgs data analysis~\cite{Khachatryan:2016vau}.
However, here we use the $Br(H\to invisible) < 28\% $ limit from a dedicated ATLAS search
as it is less model dependent.}

For the second observable, the di-photon decay rate, we consider the result from the combined fit on the signal strength in the di-photon channel~\cite{Khachatryan:2016vau}:
\begin{equation}
\frac{Br^{BSM}(H\to \gamma\gamma)}{Br^{SM}(H\to \gamma\gamma)} =\mu^{\gamma\gamma} = 1.14^{+0.38}_{-0.36}\,,
\label{eq:lhc-higgs-aa}
\end{equation}
%\comment{(MT:I have updated the reference and values above to the published version of the conference note previously used. The values change only very slightly. Note that I have doubled the errors as an estimate of the 95\% CL as Sasha did with the old results. The old values were $\mu^{\gamma\gamma} = 1.16^{+0.40}_{-0.36}$.)}\\
where we doubled the $1\sigma$ errors given in \cite{Khachatryan:2016vau} to obtain the $\mu^{\gamma\gamma}$ range at the 95\% CL.
A sufficiently light charged Higgs with sufficiently large $\lambda_3$ coupling to the SM Higgs boson,
which would bring the $H\to \gamma\gamma$ decay beyond the quoted limit, is excluded.

It should be noted that we would expect a proper 2-parameter 
fit to lead to stronger constraints that the ones we use, however the qualitative impact of the constraints should be unchanged.
For example, the partial decay width of the Higgs into  DM which is defined by
\begin{equation}
 \Gamma (H \to h_1 h_1) = \frac{1}{8\pi}\frac{\lambda_{345}^2 M_W^2}{g_W^2 M_H}\sqrt{1-4\frac{M_{h_1}^2}{M_H^2}},
\end{equation}
where $g_W$ is the weak coupling constant,  provides the following  bound on $\lambda_{345}$:
\begin{equation}
|\lambda_{345}| < 
\left(
\frac{8\pi g_W^2 \Gamma_{SM} M_H}{M_W^2 \left(\frac{1}{Br^{max}_{invis}}-1\right)\sqrt{1-4\frac{M_{h_1}^2}{M_H^2}}}	
\right)^{1/2},\label{lam345-limit-from-inv}
\end{equation}
where $Br^{max}_{invis}=0.28$ is the current bound on the maximal value of branching ratio
of the Higgs boson decay into invisible mode.
The above limit on $\lambda_{345}$ is $M_{h_1}$ dependent:
for  $M_{h_1}/M_{H} \ll 1$  it is about 0.019, while for  $M_{h_1}$ closer to the threshold,
e.g. 60 GeV, the limit on  $\lambda_{345}$ increases almost by a factor of two and reaches a value of 0.036.
In addition we have included the limit from $H\to h_2 h_2$
when $h_2$ is close in mass to $h_1$, which can be trivally done, taking into account that
$H h_2 h_2$ coupling is equal to $\tilde \lambda_{345}$ in Eq.~(\ref{tildelam345}).
%\begin{equation}
%\lambda_{Hh_2h_2}= \lambda_{345}+ \frac{2(M_{h_2}^2-M_{h_1}^2)}{v^2}.
%\label{eq:Hh2h2}
%\end{equation}
We discuss these limits in more details below, together with the Dark Matter (DM) constraints.

%set an important constraint coming from  $H \rightarrow h_1h_1$
%invisible Higgs boson decay which is restricted to be 
%\begin{equation}
%Br(H\to invisible) < 28\% 
%\label{eq:lhc-higgs-invis}
%\end{equation}
%at 95$\%$ CL as comes from ATLAS~\cite{Aad:2015txa} analysis which is comparable with 36\% limit
%from  combined CMS analysis~\cite{CMS:2015naa}\footnote{One could also limit $Br(H\to invisible)$ 
%using $Br(H\to BSM)<32\% $ at 95$\%$CL exclusion from Run1  ATLAS-CMS Higgs data analysis~\cite{ATLAS-CMS-2015-HIGGS-COMB}.
%However, here we use $Br(H\to invisible) < 28\% $ limit from the dedicated ATLAS  search
%as it is less model dependent.}.
%At the same time, the restriction  on the charged Higgs mass, $M_{h^+}$, from LHC comes from  
%\begin{equation}
%\frac{Br^{BSM}(H\to \gamma\gamma)}{Br^{SM}(H\to \gamma\gamma)} =\mu^{\gamma\gamma} = 1.16^{+40}_{-36}
%\label{eq:lhc-higgs-aa}
%\end{equation}
%limit from combined Run1 results from   ATLAS-CMS Higgs data analysis~\cite{ATLAS-CMS-2015-HIGGS-COMB}.
%A sufficiently light charged Higgs with sufficiently large $\lambda_3$ coupling to the SM Higgs boson
%which  deviate  radiative $H\to \gamma\gamma$ decay beyond the quoted limit, is excluded.
%
%This limit requires small values of $\lambda_{345}$ for the case when this decay is open i.e. for the case 
%of $M_{h_1}<M_{H}/2$. We discuss this limit in more details below, together with the 
%Dark Matter (DM) constraints.
% 
%\giac{[COMMENT: }{I think that, in principle, we should do a 2-parameter fit of the Higgs data, with only the $H\gamma\gamma$ and the $H$ to invisible couplings turned on. So, none of the fits in~\cite{ATLAS-CMS-2015-HIGGS-COMB} applies directly. I can add a few lines to comment on this. Note that a fit with less floating parameters will lead to a stronger bound. In practice, whet you did here should be fine -- unless we want to redo the fit ourselves.]} 

%%%%%%%%%%%%%%%%%%%%%%

