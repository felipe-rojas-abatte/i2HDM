\section{Introduction}

The evidence for dark matter (DM) is well-established from several independent cosmological observations, 
including galactic rotation curves, cosmic microwave background fits of the WMAP and PLANCK data, 
gravitational lensing, large scale structure of the Universe, as well as interacting galaxy clusters such as the Bullet Cluster.
Despite these large-scale evidences, the microscopic nature of the DM particles remains unknown,
since no experiment so far has been able to claim their detection in the laboratory and probe their properties.
Potentially, DM can be produced at the LHC and probed in the DM direct detection (DD) underground experiments.
The fundamental importance and vast experimental opportunities make the search for and investigation of DM 
one of the key goals in astroparticle physics and high energy physics (HEP), worthy of the intense efforts undertaken by the physics community.


At the other end of the length scale, the Standard Model (SM) of particle physics recently demonstrated its vitality once again.
The scalar boson with mass $m_H \approx 125$ GeV found at the LHC \cite{Chatrchyan:2012xdj,Aad:2012tfa} 
closely resembles, in all its manifestations,
the SM Higgs boson. Since the SM cannot be the ultimate theory, many constructions beyond the SM (BSM) 
have been put forth, at different levels of sophistication. Yet, without direct experimental confirmation,
none of them can be named the true theory beyond the SM.

One way the particle theory community can respond to this situation is to propose simple, fully calculable, renormalizable BSM models 
with viable DM candidates. We do not know yet which  of these models (if any) corresponds to reality,
but all models of this kind offer an excellent opportunity to gain insight into the intricate interplay among various
astrophysical and collider constraints. We call here these models Minimal Consistent Dark Matter (MCDM) models.
MCDM models which can be  viewed as  toy models, are self-consistent and can be easily be incorporated into a bigger BSM model.
Because of these attractive features, MCDM models can be considered as the next step beyond DM Effective Field Theory (EFT) (see e.g. \cite{Fox:2011pm,Rajaraman:2011wf,Goodman:2010ku,Bai:2010hh,Beltran:2010ww,Goodman:2010yf,Fox:2011fx,Shoemaker:2011vi,Fox:2012ru,Haisch:2012kf,Busoni:2013lha,Busoni2014a,Belyaev:2016pxe}) and simplified DM models (see e.g. \cite{Buchmueller:2013dya,Busoni:2014sya,Busoni:2014haa,Buchmueller:2014yoa,Buckley:2014fba,Abdallah:2015ter,Abdallah:2014hon,Abercrombie:2015wmb}).


In this paper, we explore, in the light of the recent collider, astroparticle and DD DM experimental  data, 
the inert Two-Higgs Doublet Model (i2HDM), also known as the Inert doublet model.
This model is easily doable with analytic calculations, its parameter space is relatively small and can be strongly 
constrained by the present and future data.
The model leads to a variety of collider signatures, and, in spite of many years of investigation, not all of them have yet been fully and properly explored.
It is the goal of the present paper to investigate in fine detail the present constraints and the impact of the future LHC 
and DD DM data on the parameter space of this model.

%%%%%%%%%%%%%%          basic definitions          %%%%%%%%%%%%%%%%%

The i2HDM \cite{Deshpande:1977rw,Ma:2006km,Barbieri:2006dq,LopezHonorez:2006gr} is a minimalistic extension of the SM 
with a second scalar doublet $\phi_2$ possessing the same quantum numbers as the SM Higgs doublet $\phi_1$
but with no direct coupling to fermions (the inert doublet). 
This construction is protected by the discrete $Z_2$ symmetry under which $\phi_2$ is odd and all the other fields are even. 
The scalar Lagrangian is
  \begin{equation}
  \mathcal{L} = |D_{\mu}\phi_1|^2 + |D_{\mu}\phi_2|^2 -V(\phi_1,\phi_2)
  \textrm{.}
  \end{equation}
with the potential $V$ containing all scalar interactions compatible with the $Z_2$ symmetry:
\begin{eqnarray}
  V &=& -m_1^2 (\phi_1^{\dagger}\phi_1) - m_2^2 (\phi_2^{\dagger}\phi_2) + \lambda_1 (\phi_1^{\dagger}\phi_1)^2 + \lambda_2 (\phi_2^{\dagger}\phi_2)^2    \nonumber  \\
  &+&  \lambda_3(\phi_1^{\dagger}\phi_1)(\phi_2^{\dagger}\phi_2) 
  + \lambda_4(\phi_2^{\dagger}\phi_1)(\phi_1^{\dagger}\phi_2) 
  + \frac{\lambda_5}{2}\left[(\phi_1^{\dagger}\phi_2)^2 + (\phi_2^{\dagger}\phi_1)^2 \right]\,.\label{potential}
\end{eqnarray}
All free parameters here are real,\footnote{Even if we started with a complex $\lambda_5$, 
we could redefine the second doublet via a global phase rotation, which would render $\lambda_5$ real
without affecting any other part of the Lagrangian.} which precludes the $CP$-violation in the scalar sector.
There is a large part of the parameter space, in which only the first, SM-like doublet, 
acquires the vacuum expectation value (vev).
In the notation $\langle\phi_i^0\rangle = v_i/\sqrt{2}$, this inert minimum corresponds to
$v_1 = v$, $v_2 = 0$.
In the unitary gauge, the doublets are expanded near the minimum as
\begin{equation}
\phi_1=\frac{1}{\sqrt{2}}
\begin{pmatrix}
0\\
v+H 
\end{pmatrix}
  \qquad
  \phi_2= \frac{1}{\sqrt{2}}
\begin{pmatrix}
 \sqrt{2}{h^+} \\
 h_1 + ih_2
\end{pmatrix}
\end{equation}
The $Z_2$ symmetry is still conserved by the vacuum state, which forbids direct coupling of any single inert field to the SM fields
and it stabilizes the lightest inert boson against decay. 
Pairwise interactions of the inert scalars with the gauge-bosons and with the SM-like Higgs $H$ are still possible,
which gives rise to various i2HDM signatures at colliders and in the DM detection experiments.

%%%%%%%%%%%       overview of previous studies     %%%%%%%%%%%

The idea that the symmetry-protected second Higgs doublet naturally produces a scalar dark matter candidate
was first mentioned more that 30 years ago \cite{Deshpande:1977rw}.
However, the real interest in phenomenological consequences of the i2HDM woke up in mid-2000 and intensified
in the last few years.
Its simplicity, predictive power, rich yet manageable parameter space, makes it an ideal playground for checking 
its compatibility with the DM relic density, with the results of the direct and indirect DM searches,
and with collider searches of various BSM signals.

%%%%   DM related
Assuming that the lightest inert scalar is the only DM candidate,
one typically finds that the low-mass region, below about 50 GeV, is excluded 
by the relic density constraints coupled with the LHC constraints on the invisible Higgs decay 
\cite{Krawczyk:2013jta,Ilnicka:2015jba,Diaz:2015pyv}.
The funnel region, with the DM mass close to $M_H/2$, the intermediate, 100--500 GeV, 
and the high mass regions are still compatible with data and lead to interesting predictions at colliders.
Additional theoretical constraints on the parameter space and DM candidate properties
can be deduced from assumptions of full stability of the i2HDM up to the PLANCK scale
 \cite{Chakrabarty:2015yia,Khan:2015ipa} or of multi-doublet Higgs inflation \cite{Gong:2012ri}.
The i2HDM can also produce signals for direct \cite{Arina:2009um} and indirect DM search experiments via heavy inert scalar annihilation,
which can be detectable via $\gamma$-rays \cite{Modak:2015uda,Queiroz:2015utg,Garcia-Cely:2015khw}
or via its neutrino \cite{Agrawal:2008xz,Andreas:2009hj} and cosmic-ray signals \cite{Nezri:2009jd}.

%%%%   cosmological consequences 
The i2HDM can also have interesting cosmological consequences.
Being an example of 2HDM, it possesses a rich vacuum structure, which evolves at high temperatures 
\cite{Turok:1991uc,Cottingham:1995cj,Ginzburg:2009dp}.
This opens up the possibility within i2HDM that the early Universe, while cooling down, 
went through a sequence of phase transitions including strong first-order phase transitions 
\cite{Ginzburg:2010wa,Chowdhury:2011ga,Borah:2012pu,Gil:2012ya,Dorsch:2013wja,Cline:2013bln,Blinov:2015vma}.
Such analyses are capable of restricting the parameter space;
for example, the recent study \cite{Blinov:2015vma} showed that combining the strong 
first-order phase transition with other astroparticle and collider constraints gives preference to the funnel region.

%%%%   collider consequences
There has also been a number of studies on collider signatures of the i2HDM. 
They focus either on specific processes such as SM-like Higgs decays to $\gamma\gamma$ and $\gamma Z$ 
\cite{Arhrib:2012ia,Swiezewska:2012eh,Krawczyk:2013jta,Krawczyk:2013pea},
multi-lepton plus missing transverse momentum ($\MET{}$) \cite{Miao:2010rg,Gustafsson:2012aj,Hashemi:2016wup}
with as many as five leptons \cite{Datta:2016nfz},
dijet$+\MET{}$ \cite{Poulose:2016lvz} and dileptons accompanied with dijets \cite{Hashemi:2016wup}.
Other works present combined analyses of astroparticle and collider constraints 
\cite{Goudelis:2013uca,Arhrib:2013ela,Ilnicka:2015jba,Alves:2016bib,Datta:2016nfz}.
Comparing the i2HDM predictions with the electroweak precision data, 
the measured SM-like Higgs properties, the non-observation
of long-lived charged particles and other exotic signals, and finally the astroparticle
observations, allows one to significantly restrict the i2HDM parameter space.
%%% 
The recent analysis \cite{Ilnicka:2015jba} gave a detailed account of these constraints.
For specific benchmark points or benchmark planes in the surviving parameter space, 
it predicted the cross section of pair production of inert scalars followed by various modes of their decay.
As for the specific signatures of the i2HDM at the LHC, dileptons and mono-$Z$ signals were mentioned.
An earlier analysis \cite{Arhrib:2013ela} investigated multilepton, multijet, mono-$Z$,
and several channels for the mono-jet with large $\MET{}$.
Ref.~\cite{Goudelis:2013uca} took into account one-loop corrections to the masses
and, for a part of the numerical scans, included the additional theoretical constraint that 
the perturbativity and stability be satisfied up to a large scale $\Lambda$.
The version of i2HDM equipped with Peccei-Quinn $U(1)$ symmetry spontaneously broken to $Z_2$ 
was investigated in \cite{Alves:2016bib}. Here, dark matter acquires a second component, the axion, 
which changes the DM phenomenology.
It is also possible to hunt for i2HDM at the future colliders, via searching for new scalars
and reconstructing the potential \cite{Aoki:2013lhm} or by accurately measuring the SM-like Higgs couplings
and deducing patterns of the deviations from the SM \cite{Kanemura:2016sos}.

In the present work, to these many 
studies on the i2HDM, we add:
\begin{itemize}
\item detailed combined analysis of the i2HDM model in its full five-dimensional (5D)  parameter space, taking
into account perturbativity and unitarity,
LEP and electroweak precision data, Higgs data from the LHC, DM relic density, direct/indirect DM detection 
complemented by realistic (beyond-the-parton-level) LHC mono-jet analysis at the LHC;
\item
quantitative exploration of the surviving regions of parameters, including very fine details  and 
qualitatively new region not seen in previous studies,
which is enabled by our extensive numerical scans;
%%%
\item
%%% modified Igor
a combination of different processes giving the LHC mono-jet signatures: 
those with direct DM pair production and those with associate production of DM with another scalar with a close mass
from the inert multiplet;
\item 
implication of experimental LHC studies on disappearing
charged tracks relevant to high ($\simeq 500$ GeV) DM mass region;
\item 
separate, equally detailed analyses for the assumptions of the DM relic density being fitted to the PLANCK results
or under-abundant, allowing thus for additional allowed regions of the parameter space.
\end{itemize}
All these points above are in close focus of the present paper where we have performed a comprehensive  
scan and study of the  full parameter space of the i2HDM model. 
In addition we have performed an independent implementation and validation of the model in two gauges 
including Higgs-gluon-gluon and Higgs-photon-photon effective couplings, and we made it public together with the LanHEP model source.

The paper is organised as follows. 
In Sect.~2 we discuss the i2HDM model parameter space, implementation, 
theoretical constraints as well as constraints from LEP and electroweak precision data.
In Sect.~3 we discuss results of a comprehensive scan of the i2HDM parameter space
and combined constraints considering both the cases when the relic density is ``just right"
and agrees with the PLANCK results and when it is under-abundant.
In this section we also present the reach of  LHC studies in the high DM mass region using
results on disappearing charged tracks.
In Sect.~4 we present results on future projections of the LHC and DM DD experiments
in combination with all previous constraints. Finally, in Sect.~5 we draw our conclusions.
 
