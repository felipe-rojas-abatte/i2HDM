\section{Concluding remarks}
The i2HDM is a clear example of a minimal consistent DM model which is
very well motivated by theoretical considerations.
At the same time this model could provide
 mono-jet, mono-$Z$, mono-Higgs and VBF+$\MET{}$
signatures at the LHC
complemented by
signals in direct and indirect DM search experiments.

The model is implemented into the CalcHEP and micrOMEGAs packages
and is publicly available at the HEPMDB database together with the
LanHEP model source. It is ready for further
exploration in the context of the LHC, relic density and DM direct detection.

In this paper we have performed 
detailed analysis of the   constraints in the full 5D  parameter space of the i2HDM from perturbativity, unitarity,
electroweak precision data, Higgs data from the LHC,
DM relic density, direct/indirect DM detection and the LHC mono-jet analysis as well as 
implications of experimental LHC studies on disappearing
charged tracks relevant to high DM mass region.
The LHC mono-jet analysis for the i2HDM model has been performed at the fast detector simulation level
and provides new results together with limits from disappearing
charged tracks at the LHC.
Our results on non-LHC constraints are summarised in Figs.~\ref{fig:dm-i2hdm}--\ref{fig:dm-i2hdm-relic-small}
which show the effect of consequent application of constraints
from:  Cut-1) vacuum stability, perturbativity and unitarity;
Cut-2) electroweak precision data, LEP constraints and the LHC Higgs data;
Cut-3) relic density constraints, and  Cut-4) constraints from LUX on DM from direct detection.
In this paper we have explored for the first time 
the parameter space where 
DM from the i2HDM is underabundant implying an additional source of DM,
using above constraints complemented by the  collider searches.
We have also explored  the parameter space in which the DM candidate of i2HDM represents 
100\% of the total DM budget of the Universe.
We found that the parameter space with  
$M_{h_1},M_{h_2}<45~\mbox{GeV} 
\mbox{\ or\ } M_{h^+}<70~\mbox{GeV}$
is completely excluded, confirming the first limit found previously
complemented the second one found in this study.

Though in general the parameter space of the i2HDM is 5-dimensional,
the parameter space relevant to the LHC mono-jet signature is only 1 or 2 dimensional,
so the model can be easily explored at the LHC.
There are two qualitatively different and complementary channels in mono-jet searches:
$pp\to h_1 h_1j$ and $pp\to h_1 h_2 j$,
with the second one being relevant to mono-jet signature
when the mass gap between $h_2$ and $h_1$
is of the order of a few GeV.
In the case of $h_1 h_2$ degeneracy, the rate for $pp\to h_1 h_1j$
will be effectively doubled since $g_{Hh_2 h_2} = g_{Hh_1h_1}$, see Eq.~(\ref{tildelam345}),
and this can be easily taken into accounts for the estimation of constraints
in the respective region of the parameter space.
For a fixed $M_{h_1}$, the strength of the first  process 
depends only on $\lambda_{345}$ because the Higgs boson is the only mediator,
while the  strength of the second  process 
is fixed by the weak coupling since the $Z$-boson is the  only mediator
for this process.
The last process is important to cover the $h_1 h_2$ co-annihilation region
available for 54 GeV $< M_{h_1} <74$ GeV,
where the relic density agrees with the PLANCK data.
The results on this process and on this region are new to our best knowledge.
Therefore these two processes complement each other in covering the parameter space:
for large values of $\lambda_{345}$, $pp\to h_1 h_1j$ would be the dominant LHC
signature, while for small or vanishing values of $\lambda_{345}$, the
$pp\to h_1 h_2 j$ process will cover additional parameter space
as demonstrated in Fig.~\ref{collider-XENON1T-constraint}--\ref{fig:dm-i2hdm-LHC-DD}.

Talking about quantitative results,
the LHC has rather limited potential to probe $M_{h_1}$
with the mono-jet signature. Even for the projected luminosity of 3 ab$^{-1}$,
we have found that the LHC could set a limit on $M_{h_1}$  up to 83 GeV
from the $pp\to h_1 h_1j$ process with the maximal value allowed for  $\lambda_{345}$
and  only up to 55 GeV from  $pp\to h_1 h_2j$
for any value of  $\lambda_{345}$, covering just the tip of the $h_1 h_2$ co-annihilation
region. Such a weak sensitivity of the LHC is related to
the similarity between the shapes of the \MET{} distribution of the dominant $Zj\to \nu\nu j$ background
and that of the signal which has the same $Z$-boson mediator, while the DM mass is not very different from $M_Z/2$,
which as shown in \cite{Belyaev:2016pxe} is the reason for such a similarity in \MET{} shape.
At the same time, the potential of the LHC using a search for disappearing charged tracks
is quite impressive  in probing  $M_{h_1}$ masses up to about 500 GeV
already at 8 TeV with 19.5 fb$^{-1}$ luminosity
as we have found in our study.


We have also explored the projected potential of XENON1T to probe the i2HDM
parameter space and have found that it is quite impressive,
confirming results of previous studies.
In our study we have presented  ``absolutely allowed"  and ``absolutely excluded" points
in different projections of the i2HDM 5D space demonstrating different features of the models
and the potential of current and future experiments.
In general, DM DD experiments and collider searches complement each other:
the $pp\to h_1 h_1j$ process covers in the region with large $\lambda_{345}$ coupling
where DM DD rates are low because of the low relic density re-scaling,
while the $pp\to h_1 h_2j$ process is sensitive to the parameter space with low $\lambda_{345}$
where DM DD rates are low because of the low rate of DM scattering off the nuclei.
