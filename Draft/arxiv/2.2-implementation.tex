\subsection{Model implementation}
We have implemented the i2HDM into the CalcHEP package~\cite{CALCHEP}
with the help of the LanHEP program~\cite{Semenov:1998eb,Semenov:2008jy}
for automatic Feynman rules derivation. The effective $Hgg$ and $H\gamma\gamma$ vertices were included and the model was cross-checked in two different gauges
to ensure a correct, gauge invariant implementation.
It is publicly available at the
High Energy Physics Model Data-Base (HEPMDB) \cite{Brooijmans:2012yi} at
\url{http://hepmdb.soton.ac.uk/hepmdb:0715.0187}
together with the LanHEP source of the model.
The model is implemented in terms of the five independent parameters
defined in Eq.~(\ref{eq:model-parameters}), consisting of three physical masses
and two couplings.
We found this choice the most convenient for exploration of i2HDM phenomenology
and constraints of its parameter space.
We should stress that the $M_{h_1}$ and $M_{h_2}$ parameters 
conveniently  define the mass order of the two neutral inert states 
$independently$ of their $CP$ properties. 
This choice is especially convenient and relevant for collider
phenomenology since, as we discussed above, one can not assign (or determine) the CP parity of each neutral inert scalar.

To explore the phenomenology of the i2HDM we need to consider other constraints on its parameter space
in addition to those coming from vacuum stability which we discussed above.

